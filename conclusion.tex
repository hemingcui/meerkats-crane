\section{Conclusion} \label{sec:conclusion}
We have presented \xxx, a \smr system that 
transparently replicates general server programs without 
requiring server developers' intervention. It provides a new state machine 
interface compatible to socket API, and it leverages 
deterministic multithreading to enforce the same schedules for a multithreaded 
server program across replicas. \xxx creates a \timealgo technique to 
efficiently enforce consistent logical times on admitting network requests 
across replicas.

Evaluation on \nprog widely used server programs shows that 
\xxx is easy to use, has moderate overhead, and provides practical recovery 
support. \xxx has the potential to expand the adoption of \smr and to provide 
transparent fault-tolerance support for general server programs. \xxx's 
source code is at \github.

\section{Recommendation} \label{sec:recommendation}
TBD.
% P11: Contributions, conceptual and engineering.
% Our key conceptual contribution is the idea of transparent \smr for
% general programs, which has the potential to expand \smr's
% adoption and improve availability of many systems.  This idea also applies to 
% other replication concepts (\eg, byzantine fault 
% tolerance~\cite{pbft:osdi99,zyzzyva:sosp07}). This idea has other broad 
% applications as well (\S\ref{sec:app}). Our engineering contributions 
% include the \xxx system and its evaluation on widely used server
% programs. 
% 
% All \xxx's source code (including a standalone, libevent-based
% \paxos implementation), benchmarks, and evaluation results are available
% at \github.
